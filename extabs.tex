\documentclass{article}
\begin{document}

\section{Pitch 1}
\textbf{The Idea:  Language as an improvement to existing CrashSim
Tool}\break

CrashSimulator has already proven effective in finding bugs but there is
room for improvement.   One area where using the tool could be simplified is in
writing the checkers and mutators.  To this end, we developed a domain specific
language based on enhanced regular expressions to simplify this process.
Using this domain specific language CrashSimulator’s developers and users
implemented YYY new checkers and mutators and used them to identify ZZZ new bugs
in real world applications.

\subsection{Contributions}
\begin{enumerate}
\item{Design for domain specific language for describing sequences of
    system calls.}
\item{Implementation of this design as a new way to write checkers and mutators for
    the existing CrashSimulator tool}
\item{Demonstration that these improvements have enhanced CrashSimulator’s ability
    to find bugs}
\item{User study evidence that users find this language easier to grasp and more efficient to
    use than CrashSimulator’s previous way of describing checkers and mutators}
\end{enumerate}

\section{Pitch 2}
\textbf{The Idea: We have implemented a language for describing system
call sequences and proved its usefulness by using it in CrashSim}\break

Many instances of bugs, security problems and malicious activity are
visible in an application's system call activity.  We can take advantage of
this fact to identify these problems by identifying specific,
problematic system call sequences.  To do this, we need a way to describe
these sequences.  In this work we show that these sequences can be modeled
using finite automata with registers and offer a way to construct these
automata using {\tt RELANG}, a new domain specific
language similar to regular expressions with back
references.  To illustrate the effectiveness of this approach we
used {\tt RELANG} to describe the system call sequences used by
CrashSimulator's checkers and mutators.  CrashSimulator was able to
identify YYY previously-undiscovered bugs in major applications using
checkers and mutators written in {\tt RELANG}.  Additionally, we conducted a
user study that found {\tt RELANG} was both easier to learn and more
efficient to work with than CrashSimulator's previous Python-based checker
and mutator description strategy.


\section{Contributions}
\begin{enumerate}
\item{Demonstration that system call sequences can be recognized by finite
    automata with registers}
\item{Design for domain specific language for describing these automata in
    as regular expressions with back references}
\item{Implementation of this language using ZZZ finite automata
    algorithm(s)}
\item{Use of language to describe checkers and mutators in CrashSimulator
    that resulted in the identification of YYY new bugs.}
\item{Results from a user study using the above implementation that show the
language is easier for user to learn and use.}
\end{enumerate}

\section{Related Work}

\begin{enumerate}
    \item{\textit{https://www.cs.unm.edu/~forrest/publications/jcs-sequences-of-system-calls-98.pdf}\break
        Uses system call sequences (without examining parameters etc, specific concrete
        sequences only) to detect malicious behavior}
    \item{\textit{https://pdfs.semanticscholar.org/df3b/6e275f8e1885e7bc922afb19b012dba84635.pdf}\break
Uses non-sequenced system call data to detect anomalous behavior
        statistically}
    \item{\textit{https://arxiv.org/pdf/1707.03821.pdf}\break
Classifies misbehavior by counts of system calls in a given period of time using
        machine learning}
    \item{\textit{https://arxiv.org/pdf/1002.0432.pdf}\break
        Proposes a regular expression-like language for system call “motifs”}
\end{enumerate}

\end{document}
